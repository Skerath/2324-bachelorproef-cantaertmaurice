%%=============================================================================
%% Inleiding
%%=============================================================================

% context, achtergrond
% afbakenen van het onderwerp
% verantwoording van het onderwerp, methodologie
% probleemstelling
% onderzoeksdoelstelling
% onderzoeksvraag

\chapter{\IfLanguageName{dutch}{Inleiding}{Introduction}}
\label{ch:inleiding}
Sport kent een prominente rol in de levensstijl van Vlamingen, zo blijkt sterk in de statistieken van~\textcite{StatistiekVlaanderen2023}.
Hoewel sportparticipatie van Vlamingen hoog scoort met een gemiddelde van een op vier Vlamingen die (bijna) dagelijks een sport beoefent, kan hetzelfde niet gezegd worden over verhouding beschikbare trainers.
Ondanks de stijging van het aantal gediplomeerde trainers blijkt in het jaarverslag van~\textcite{SportVlaanderen2023} blijkt dat sportclubs blijven kampen met een trainerstekort.
Sterker nog, het komt naar voren dat gemiddeld 26 procent van de trainers jonger dan 30 jaar afhaken, waardoor de verhouding van één trainer op twintig sporters geen positieve evolutie kent.

Het is duidelijk dat er niet alleen werk gemaakt moet worden om meer coaches op te leiden, maar ook om tewerkgestelde trainers bij te staan in hun takenpakket.
Met de voortdurende verbeteringen binnen de artifici\"ele intelligentiesector en de toename in capaciteiten van generatieve AI ontstaan mogelijkheden om de werkdruk van trainers te verlichten.
Deze paper richt zich erop een proof-of-concept Android-app te ontwikkelen om sporters op weg te helpen met behulp van generatieve AI en objectherkenning.
Trainers kunnen vervolgens bijsturen waar nodig, waarmee de werkdruk om startende sporters bij te staan verkleind kan worden.

%%---------- Probleemstelling -------------------------------------------------
\section{\IfLanguageName{dutch}{Probleemstelling}{Problem Statement}}
\label{sec:probleemstelling}
Jordi Van Bossuyt is een jonge zelfstandige personal trainer en geeft momenteel training aan cadetten in de Koninklijke Atletiekclub Eendracht Aalst.
Met een interesse voor een meer gestroomlijnde manier om startende sporters bij te staan komt het idee van een combinatie van een mobiele app met generatieve AI aan bod.
Concreet is het doel om een app te ontwikkelen waarbij er minimale frictie moet zijn om aan de slag te gaan met sportmateriaal.
Gebruikers horen de mogelijkheid te krijgen om sportgerelateerde objecten en omgevingen zoals fitnessinstrumenten of looppistes in te scannen met de camera van de telefoon en hiervoor suggesties te krijgen.
Met een proof-of-concept moet aangetoond kunnen worden dat dit mogelijk is door middel van een vooraf-getrainde dataset en bestaande generatieve AI-modellen om zelf suggestieve trainingsschema's te genereren.

%%---------- Onderzoeksvragen -------------------------------------------------
\section{\IfLanguageName{dutch}{Onderzoeksvraag}{Research question}}
\label{sec:onderzoeksvraag}
Hoe kan een bestaande kunstmatige intelligentieplatform gebruikt worden om aan nauwkeurige objectherkenning te doen?
Kan het platform vervolgens suggesties van activiteiten voorstellen aan de gebruiker?
Op welke manier kan de trainer inspelen op het de suggesties die gegenereerd worden door de app?

%%---------- Onderzoeksdoelstellingen -----------------------------------------
\section{\IfLanguageName{dutch}{Onderzoeksdoelstelling}{Research objective}}
\label{sec:onderzoeksdoelstelling}
Deze bachelorproef zal zich eerst richten op het onderzoeken van de manier waarop aan objectherkenning en het suggereren van oefeningen kan gedaan worden in de vorm van een literatuurstudie.
Hierin komt tevens ook het trainen van datasets in context van machine en deep learning aan bod om een beter begrip te krijgen op de manier waarop kunstmatige intelligentiemodellen getraind en gebruikt worden.
Met dit basis begrip zal een proof-of-concept uitgewerkt worden om de haalbaarheid van een mobiele app ter ondersteuning van trainers te illustreren.

De proof-of-concept moet hierbij voldoen aan enkele criteria van de opdrachtgever:
\begin{itemize}
    \item De app hoort op een kostenefficiënte manier ontwikkeld te worden, waarbij er geen vereiste is om zelf datasets te trainen.
    \item De gebruikte technologieën horen zo dicht mogelijk bij elkaar aan te sluiten en met elkaar te integreren, er is dus één platform voor zowel de personal trainer als de cliënt.
    \item Suggesties moeten gegenereerd kunnen worden zonder de input van trainers, maar met de mogelijkheid voor trainers om deze suggesties te kunnen bijsturen.
    \item Gebruikers horen zo weinig mogelijk frictie te ervaren bij het inzenden van foto's en het ontvangen van suggesties.
\end{itemize}

%%---------- Opzet van de bachelorproef ---------------------------------------
\section{\IfLanguageName{dutch}{Opzet van deze bachelorproef}{Structure of this bachelor thesis}}
\label{sec:opzet-bachelorproef}
De rest van deze bachelorproef is als volgt opgebouwd:

In Hoofdstuk~\ref{ch:stand-van-zaken} wordt een overzicht gegeven van de stand van zaken binnen het onderzoeksdomein, op basis van een literatuurstudie.

In Hoofdstuk~\ref{ch:methodologie} wordt de methodologie toegelicht en worden de gebruikte onderzoekstechnieken besproken om een antwoord te kunnen formuleren op de onderzoeksvragen.
Ook de visie van het technisch luik komt hier aan bod.

In Hoofdstuk~\ref{ch:shortlist} worden de gekozen technologie\"en en platformen toegelicht die gebruikt zullen worden bij het ontwikkelen van de proof-of-concept.

In Hoofdstuk~\ref{ch:proof-of-concept} wordt de uitwerking van de proof-of-concept uitgelegd samen met een demonstratie van hoe de app gebruikt kan worden.

In Hoofdstuk~\ref{ch:conclusie}, tenslotte, wordt de conclusie gegeven en een antwoord geformuleerd op de onderzoeksvragen.
Daarbij wordt ook een aanzet gegeven voor toekomstig onderzoek binnen dit domein.