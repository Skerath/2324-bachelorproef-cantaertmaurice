%%=============================================================================
%% Methodologie
%%=============================================================================

\chapter{\IfLanguageName{dutch}{Methodologie}{Methodology}}
\label{ch:methodologie}
Naast de stand van zaken vindt er zich ook een technisch aspect plaats in de vorm van een proof-of-concept.
Dit hoofdstuk verduidelijkt het plan van aanpak van dit luik.

\section{Requirements-analyse}
\label{sec:requirementsanalyse}
Het doel van de requirements-analyse is om een lijst van vereisten op te stellen die de selectie van tools en de uitwerking van de proof-of-concept zullen aftoetsen.
Om een beter begrip te krijgen over de wensen van personal trainers, en in het bijzonder de opdrachtgever van deze paper, vindt er zich een korte interview plaats met de co-promotor.
Hieruit volgt een beter begrip voor de werking van personal training, de verwachtingen van cli\"enten en de workflow van coaches.
Het resultaat hiervan is terug te vinden in de~\nameref{sec:onderzoeksdoelstelling}.

\section{Selectie van tools}
\label{sec:selectie-van-tools}
Na het bepalen van de vereisten volgt een opstelling van technologie\"en en platformen die gebruikt zullen worden bij het ontwikkelen van de proof-of-concept.
In deze fase volgt de redenering waarom welke tools gekozen werden zoals afgetoetst met de vereisten.
Hiermee kan een volgend gesprek met de opdrachtgever plaatsvinden om te verifi\"eren of aan de verwachtingen voldaan zal worden.
Met een duidelijk begrip van welke tools gebruikt zullen worden kan de ontwikkeling van de proof-of-concept van start gaan.

\section{Proof-of-concept}
\label{sec:proof-of-concept2}
Eerst volgt een opzetting van de geselecteerde tools met documentatie om de werking te verduidelijken.
Vervolgens worden de gevraagde functionaliteiten ge\"{\i}mplementeerd, belangrijk hierbij is het regelmatig vragen van feedback van de co-promotor.
Daarna volgt een demonstratie van het experiment om de ondervindingen en resultaten mee te delen.
Als de opdrachtgever een positieve reactie geeft komen enkele optimalisaties aan bod, waaronder het documenteren van de gebruikte hardware en het schrijven van een script om de reproduceerbaarheid van het experiment te bevorderen.

\section{Conclusie en vooruitzichten}
\label{sec:conclusie-en-vooruitzichten}
Tot slot volgt een korte conclusie met een terugblik op de resultaten ter antwoord op de gestelde onderzoeksvragen.
Onduidelijkheden komen hierbij aan het bod, waarop enkele vooruitzichten en uitnodigingen voor verder onderzoek plaatsvinden.