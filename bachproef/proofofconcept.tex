%%=============================================================================
%% Proof of Concept
%%=============================================================================

\chapter{\IfLanguageName{dutch}{Proof-of-concept}{Proof-of-concept}}
\label{ch:proof-of-concept}
Na de~\nameref{ch:shortlist} kan de proof-of-concept uitgewerkt worden.
De uitgewerkte code kan teruggevonden worden in deze GitHub repository(TODO). % TODO github url

\section{Visie}
\label{sec:visie}
Dit onderdeel richt zich op het defini\"eren van de visie achter de uitgewerkte proof-of-concept.
De aanhaling van de verwachte resultaten, diens doelpubliek en de voordelen van het gebruik ervan benadrukken de toegevoegde waarden van het technisch luik.
Ten slotte komen de innovatieve aspecten aan bod.

\subsection{Beoogde resultaat}
\label{subsec:doel-van-de-proof-of-concept}
% een android app met achterliggende api
TODO

\subsection{Doelpubliek}
\label{subsec:doelpubliek}
% personal trainers zoals de co-promotor
TODO

\subsection{Innovatie}
\label{subsec:innovatie}
% waarom is deze PoC innovatief
TODO

\subsection{Voordelen}
\label{subsec:voordelen}
% vergemakkelijkt het start proces dat personal trainers meemaken met  nieuwe clienten
TODO

\section{Proof-of-concept omgeving opzetten}
\label{sec:proof-of-concept}
Allereerst worden de omgevingen opgezet om het uitwerken van de proof-of-concept mogelijk te maken.
Hierna komt het schrijven van de logica achter de app en achterliggende service aan bod.

\subsection{Opzetten Google Gemini omgeving}
\label{subsec:aanmaken-google-gemini-profiel}
TODO

\subsection{Opzetten Quarkus back-end}
\label{subsec:opzetten-quarkus-back-end}
De achterliggende service zal gebruik maken van de ingebouwde Quarkus extensies en Dev Services, zoals gezien in hoofdstuk~\ref{ch:shortlist}.
Hoewel Quarkus vele taken voor de ontwikkelaar vergemakkelijkt vergt het enigszins wat opzetwerk.

\subsubsection{Quarkus met Kotlin/Gradle}
%Opzetten Quarkus project met Kotlin en extensies
TODO

\subsubsection{Testcontainers}
%Opzetten test containers met configuratie
TODO

\subsection{Opzetten Jetpack Compose Android-app}
\label{subsec:opzetten-jetpack-compose-android-app}
TODO

\section{Reproduceren van de proof-of-concept}
\label{sec:reproduceren-van-de-proof-of-concept}
%Uitleggen waarom dit belangrijk is (zie cursus RM)
TODO

\subsection{Gebruikte hardware en software}
\label{subsec:gebruikte-hardware}
TODO

\subsection{Automatisatie script}
\label{subsec:automatisatie-script}
%Docker script, jar file vanuit gradle builden enz
TODO
