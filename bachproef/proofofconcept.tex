%%=============================================================================
%% Proof of Concept
%%=============================================================================

\chapter{\IfLanguageName{dutch}{Proof-of-concept}{Proof-of-concept}}
\label{ch:proof-of-concept}
Na de~\nameref{ch:shortlist} kan de proof-of-concept uitgewerkt worden.
Allereerst worden de omgevingen opgezet om het uitwerken van de proof-of-concept mogelijk te maken.
Hierna komt het schrijven van de logica achter de app en achterliggende service aan bod.
Ten slotte worden enkele belangrijke details rond reproduceerbaarheid aangehaald.

De uitgewerkte broncode kan teruggevonden worden in deze GitHub repository(TODO). % TODO github url

\section{Omgevingen opzetten}
\label{sec:omgevingen-opzetten}
De drie lagen van de~\nameref{subsubsec:architectuur} kennen enkele configuratiestappen tijdens het opzetten, de documentatie daarvan volgt hieronder.

\subsection{Opzetten Google Gemini omgeving}
\label{subsec:opzetten-google-gemini-omgeving}
%Hier wordt het AI-platform opgezet om objectherkenning en suggereren van fitnessschema's mogelijk te maken TODO
TODO

\subsection{Opzetten Quarkus service}
\label{subsec:opzetten-quarkus-service}
De achterliggende service zal gebruik maken van de ingebouwde Quarkus extensies en Dev Services, zoals gezien in hoofdstuk~\ref{ch:shortlist}.
Hoewel Quarkus vele taken voor de ontwikkelaar vergemakkelijkt vergt het enigszins wat opzetwerk.

\subsubsection{Quarkus met Kotlin/Gradle}
%Opzetten Quarkus project met Kotlin en extensies TODO
TODO

\subsubsection{Testcontainers en Docker}
%Opzetten test containers met configuratie TODO
TODO

\subsection{Opzetten Jetpack Compose Android-app}
\label{subsec:opzetten-jetpack-compose-android-app}
Ook Jetpack Compose kent enigszins wat opzetwerk, zoals het verbinden met de achterliggende dienst en het vragen van permissies  % TODO

\subsubsection{Communicatie met de achterliggende service}
TODO communicatie met API opzetten % TODO

\subsubsection{Toestemming vragen aan de gebruiker}
TODO opzetten systeem om toestemming te vragen om camera te gebruiken % TODO

\subsection{Reproduceren van de proof-of-concept}
\label{subsec:reproduceren-van-de-proof-of-concept}
%Uitleggen waarom dit belangrijk is (zie cursus RM) TODO
TODO uitleggen waarom reproduceren belangrijk is en wat er gedaan is om dit mogelijk te maken

\subsubsection{Gebruikte hardware en software}
% TODO
TODO verklaren welke software nodig is om de PoC te reproduceren en op welke hardware de testen zijn uitgevoerd

\subsubsection{Automatisatie script}
%Docker script, jar file vanuit gradle builden enz TODO
TODO verklaren welke script gebruikt moet worden om te reproduceren

\section{Uitwerking van de proof-of-concept}
\label{sec:uitwerking-van-de-proof-of-concept}
% TODO
TODO uitwerking van code met genomen keuzes en toelichting

\section{Resultaten}
\label{sec:resultaten}
% TODO
TODO demonstratie \& screenshots vanuit het perspectief van de gebruiker en de trainer