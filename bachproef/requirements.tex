%%=============================================================================
%% Requirements
%%=============================================================================

\chapter{\IfLanguageName{dutch}{Vereisten van de opdrachtgever}{Requirements of the co-supervisor}}
\label{ch:requirements}
Om aan een goede selectie van tools te komen moeten we eerst kijken naar de verwachtingen van de opdrachtgever\ldots
Hier wordt vervolgens rekening mee gehouden tijdens de selectie van tools (zie hoofdstuk TODO) en de uitwerking van de proof of concept (zie hoofdstuk TODO)

\section{Cost efficiëntie van de applicatie}
\label{sec:cost-efficientie-van-de-applicatie}
Haalbaarheid/mogelijkheid om pre trained data te gebruiken voor het detecteren van toestellen

\section{Beperkte, reproduceerbare proof-of-concept}
\label{sec:beperkte-reproduceerbare-proof-of-concept}
Bij 1 fitness apparaat nauwkeurig kunnen toepassen (beperkte proof of concept) door middel van een Android-app.
Scripts en uitleg over gebruikte hardware en platformen om te kunnen reproduceren.

\section{Suggesties door middel van generatieve AI}
\label{sec:suggesties-door-middel-van-generatieve-ai}
AI moet gepaste trainingsschema's kunnen voorstellen ervoor.
Niet gepersonaliseerd aan de gezondheid van de gebruiker (wel in conclusie eventueel verwijzen).
Eventueel data opslaan voor coaches om te kunnen zien.

\section{Simpele user experience}
\label{sec:simpele-user-experience}
Binnen de PT wordt het gedaan zodat de klant zo weinig mogelijk moet doen of nadenken.
De PT/app moet het zo goed mogelijk doen, dus het proces zo simpel mogelijk houden.
Enkel foto nemen en doorsturen, daarop feedback krijgen.