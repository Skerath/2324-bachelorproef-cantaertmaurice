%%=============================================================================
%% Samenvatting
%%=============================================================================

% De "abstract" of samenvatting is een kernachtige (~ 1 blz. voor een
% thesis) synthese van het document.
%
% Een goede abstract biedt een kernachtig antwoord op volgende vragen:
%
%
% LET OP! Een samenvatting is GEEN voorwoord!

%%---------- Nederlandse samenvatting -----------------------------------------
%
% Als je je bachelorproef in het Engels schrijft, moet je eerst een
% Nederlandse samenvatting invoegen. Haal daarvoor onderstaande code uit
% commentaar.
% Wie zijn bachelorproef in het Nederlands schrijft, kan dit negeren, de inhoud
% wordt niet in het document ingevoegd.

\IfLanguageName{english}{%
    \selectlanguage{dutch}
    \chapter*{Samenvatting}
    \selectlanguage{english}
}{}

%%---------- Samenvatting -----------------------------------------------------

\chapter*{\IfLanguageName{dutch}{Samenvatting}{Abstract}}
Grote taalmodellen binnen de kunstmatige intelligentie, en in het verlengde hiervan multimodale taalmodellen, brengen dag-na-dag steeds meer mogelijkheden ter tafel.
Technologische giganten zoals Google en Apple zien het potentieel ervan en bieden daarmee steeds meer functionaliteiten die versterkt zijn door deze taalmodellen.
Deze bachelorproef onderzoekt of deze recente innovaties een oplossing kan bieden in de sportsector, en met name de personal training. \\

Sportparticipatie binnen Vlaanderen scoort hoog met een gemiddelde van een op vier Vlamingen die (bijna) dagelijks een sport beoefent, echter kan hetzelfde niet gezegd worden over de verhouding beschikbare trainers.
Clubs blijven kampen met een trainerstekort, ook komt het naar voren dat gemiddeld 26 procent van jonge trainers afhaken door de werkdruk.
Hiermee ontstaat de vraag of de recente innovaties binnen artificiële intelligentie een oplossing kan bieden voor dit knelpunt:
Is het mogelijk om generatieve AI in te zetten om een deel van de verantwoordelijkheid van personal trainers over te nemen? \\

Deze bachelorproef richt zich erop te onderzoeken of een nieuw platform opgezet kan worden met behulp van bestaande taalmodellen en datasets.
Het platform hoort enerzijds de werkdruk te verminderen en anderzijds een positieve meerwaarde te bieden aan de behoeften van klanten.
Om dit te bereiken komt een literatuurstudie aan bod dat de huidige stand van zaken rond objectherkenning en generatieve AI in kaart brengt.
Alvorens een proof-of-concept uit te werken zal er zich eerst een korte requirements-analyse plaats vinden.
Op basis van de resultaten uit deze analyse kan er een selectie gemaakt worden van de gebruikte technologieën. \\

De proof-of-concept bewijst de mogelijke voordelen van kunstmatige intelligentie in de sportsector.
Vooropgestelde vereisten zoals het genereren van suggesties wordt overtroffen door constructieve en gestructureerde suggesties in Markdown formaat waar de coach verder op kan inspelen.
Ondanks de benodigde maatregelen om mogelijke foutieve inschattingen kunnen enkele foutieve metingen plaatsvinden.
Toch kan het implementeren van generatieve AI een meerwaarde bieden, zeker met het mogelijke vooruitzicht om biometrische data toe te passen om suggesties gepersonaliseerder te maken.