\chapter{\IfLanguageName{dutch}{Stand van zaken}{State of the art}}
\label{ch:stand-van-zaken}

% Tip: Begin elk hoofdstuk met een paragraaf inleiding die beschrijft hoe
% dit hoofdstuk past binnen het geheel van de bachelorproef. Geef in het
% bijzonder aan wat de link is met het vorige en volgende hoofdstuk.

% Pas na deze inleidende paragraaf komt de eerste sectiehoofding.


%---------- Objectherkenning -----------------------------------------------------------
\section{De werking van objectherkenning}\label{sec:ls-object-detectie}
Gezien objectdetectie de kern van dit onderzoek vormt, is het belangrijk om eerst een basis begrip te vormen van wat het juist inhoud.
Om dit te bereiken wordt eerst computer visie aangehaald, het vakdomein waarbinnen objectdetectie toebehoort.
Met een gepast begrip van objectherkenning en de manier waarop beeldfragmenten bijgehouden worden wordt het mogelijk om enkele toepassingen van bestaande algoritmen te analyseren.
Het volgende hoofdstuk binnen deze literatuurstudie bouwt hierop verder en legt de link met large language models (LLM), een ander vakgebied binnen de artifici\"ele intelligentie.

\subsection{Computer visie}\label{subsec:de-kern-van-objectdetectie}
Computer visie is een van de vele vakgebieden binnen de kunstmatige intelligentie en richt zich op het interpreteren van multimedia~\autocite{Moin2023}.
Met de komst van dit vakgebied ontstaat een grote waaier van mogelijkheden voor computers om bij te staan bij complexere taken zoals gezichts- en emotieherkenning, sc\`eneanalyse en objectherkenning.
Hiermee beschrijven~\textcite{Tasnim2023} dat objectdetectie een fundamentele taak omvat binnen dit vakgebied waarmee het identificeren en lokaliseren van objecten mogelijk wordt op (bewegende) beelden.
De ontwikkelingen in dit vakgebied worden vandaag steeds meer gebruikt in allerlei sectoren waaronder landbouw, zoals vermeld door~\textcite{Anand2024}, en gezondheidszorg~\autocite{Germanese2023}.

\subsection{Beeldfragmenten als data}\label{subsec:beeldfragmenten-als-data}
Concreet worden beelden digitaal opgeslagen in tweedimensionale tabellen van pixels, waarbij elke pixel kleurdata bevat.
De resolutie, of scherpheid, van een beeldfragment wordt vaak hierin uitgedrukt.
Zo kent een standaard Full HD (FHD)-computerscherm volgens de specificaties van~\textcite{VESA2013} een resolutie van 1920 pixels bij 1080 pixels.
Deze resolutie wordt volgens statistieken van~\textcite{ValveCorporation2024} gebruikt door meer dan de helft van haar gebruikers.
% TODO: foto van hardware resoluties
Hierbij kan de meeteenheid megapixels (MP) gebruikt worden om de kwaliteit van een beeldfragment te beschrijven, waarbij \'e\'en megapixel \'e\'en miljoen pixels voorstelt en kan de voorafgaande specificatie uitgedrukt worden in 2,1 megapixels.
Moderne smartphones zoals de Galaxy S24 nemen volgens~\textcite{Samsung2024} foto's aan een kwaliteit van 50 megapixels en bieden daarmee ongeveer 24 keer hogere kwaliteit dan een Full HD-computerscherm kan weergeven.

\subsection{Het verwerken van data uit multimedia}\label{subsec:het-verwerken-van-data}
In hoofdstuk 2 van zijn onderzoek beschrijft~\textcite{Olaoye2024} het proces van beeldverwerking in enkele cruciale stappen:
% TODO

\subsection{Bestaande objectherkenningsalgoritmen}\label{subsec:bestaande-algoritmen}
Na het verwerken van beeldfragmenten komt het toepassen van objectherkenning aan bod, waarvoor vele iteraties aan algoritmes geschreven zijn.
Belangrijk hierbij is om te onthouden dat elk algoritme eerst een dataset nodig heeft om te weten hoe een bepaald object eruit ziet.
Dit komt aan bod in het hoofdstuk~\nameref{sec:datasets}.
Hieronder volgen de meestgebruikte traditionele algoritmes om aan objectherkenning te doen.
Daarnaast worden enkele actuele toepassingen opgesomd en de effici\"entie van de algoritmes.

\subsubsection{Algoritme A}
Algoritme A wordt uitgelegd.

%---------- De werking van AI ----------------------------------------------------------
\section{Artificiële intelligentie}
\label{sec:ls-artificiele-intelligentie}
In het vorig hoofdstuk kwam computer visie aan bod als vakdomein binnen de artifici\"ele intelligentie.
Een ander, en met hedendaagse coverage in het nieuws wellicht bekendere, domein hierin is de toepassing van large language models (LLM).
Voorbeelden hiervan zijn GPT en haar toepassing ChatGPT. % TODO: hier op verder gaan & kijken naar voorstel of daar niets uit genomen kan worden.

%---------- Datasets generereren ------------------------------------------------------
\section{Datasets}\label{sec:datasets}
Uitleg hier over: % TODO
- hoe wordt bepaald wat een geldig beeld is om mee te trainen
- hoe worden datasets getraind
- misschien statistieken van hoe nauwkeurig
- bestaande datasets gebruiken zoals Google Gemini Vertex