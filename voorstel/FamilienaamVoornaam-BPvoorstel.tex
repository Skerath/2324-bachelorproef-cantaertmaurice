%==============================================================================
% Sjabloon onderzoeksvoorstel bachproef
%==============================================================================
% Gebaseerd op document class `hogent-article'
% zie <https://github.com/HoGentTIN/latex-hogent-article>

% Voor een voorstel in het Engels: voeg de documentclass-optie [english] toe.
% Let op: kan enkel na toestemming van de bachelorproefcoördinator!
\documentclass{hogent-article}
\usepackage{graphicx}
\graphicspath{ {images/} }

% Invoegen bibliografiebestand
\addbibresource{voorstel.bib}

% Informatie over de opleiding, het vak en soort opdracht
\studyprogramme{Professionele bachelor toegepaste informatica}
\course{Bachelorproef}
\assignmenttype{Onderzoeksvoorstel}
% Voor een voorstel in het Engels, haal de volgende 3 regels uit commentaar
% \studyprogramme{Bachelor of applied information technology}
% \course{Bachelor thesis}
% \assignmenttype{Research proposal}

\academicyear{2023-2024}
\title{Fotografische objectdetectie van fitnessapparatuur voor het genereren van een gepersonaliseerd trainingsschema}

\author{Maurice Cantaert}
\email{maurice.cantaert@student.hogent.be}

% TODO: Geef de co-promotor op
%\supervisor[Co-promotor]{In overleg (Tropos AR, \href{mailto:blabla@tropos.ar}{blabla@tropos.ar})}
\supervisor[Co-promotor]{In overleg}

% Binnen welke specialisatierichting uit 3TI situeert dit onderzoek zich?
% Kies uit deze lijst:
%
% - Mobile \& Enterprise development
% - AI \& Data Engineering
% - Functional \& Business Analysis
% - System \& Network Administrator
% - Mainframe Expert
% - Als het onderzoek niet past binnen een van deze domeinen specifieer je deze

%   zelf
%
\specialisation{Mobile \& Enterprise development}
\keywords{Machine learning, gepersonaliseerd trainingsschema, objectdetectie}

\begin{document}
    \begin{abstract}
        Vandaag de dag kent sport een prominente rol bij vele Vlamingen, dat kunnen we duidelijk aflezen uit statistieken van \textcite{StatistiekVlaanderen2023} .
        Hoewel sporten centraal staat voor gemiddeld \'e\'en op de vijf Vlamingen, kampt de sportsector met een sterk trainerstekort. \autocite{SportVlaanderen2023}
        De overheidsinstantie Sport Vlaanderen heeft reeds enkele campagnes gelanceerd om dit knelpunt te verhelpen.
        Deze bachelorproef richt zich er op om voor fitnessclubs een ondersteunende tool te ontwikkelen die haar leden kan helpen om minder te hoeven ondersteunen op trainers.
        Hiermee horen sporters simpelweg hun uitrusting te kunnen scannen, en zouden ze een gepersonaliseerd trainingsschema voorgeschoteld krijgen.
        Maar hoe wordt specifieke apparatuur juist herkend op een nauwkeurige en behulpzame manier?
        Welke methodes kunnen gebruikt worden om het gepaste trainingsschema bij de juiste soort apparatuur voor de gebruiker voor te stellen?
        Hoe leggen we de link tussen het trainingsschema en de trainer om een steunmiddel aan te bieden voor zowel de sporter als de trainer?
        Om dit voorstel te realiseren wordt allereerst een literatuurstudie uitgevoerd om de huidige stand van zaken rond het detecteren van objecten (in dit geval fitness apparatuur) in kaart te brengen.
        Hiermee ontstaat de mogelijkheid om een proof-of-concept uit te werken, gebruikmakend van een machine learning dataset, om gebruikservaringen af te toetsen.
        Verwacht wordt dat deze tool een duidelijke hulp kan bieden voor beginnende sporters.
    \end{abstract}

    \tableofcontents

% De hoofdtekst van het voorstel zit in een apart bestand, zodat het makkelijk
% kan opgenomen worden in de bijlagen van de bachelorproef zelf.
    %---------- Inleiding ---------------------------------------------------------

\section{Introductie}%
\label{sec:introductie}

Augmented reality, of kortweg AR, omvat een samensmelting van de virtuele wereld met de echte wereld door middel van een overlay.
Met haar eerste iteratie in de vorm van een head-mounted display ontwikkeld door \textcite{Sutherland1968} , kende deze technologie een opmerkelijke voorsprong op de hedendaagse smartphone.
Toch kent deze vorm van AR niet dezelfde aanneming door het algemene publiek en is tot recentelijk enkel een vorm van mobiele AR (gebruikmakend van smartphones) aanwezig op de markt.

Head-mounted AR had vooral te lijden aan grotere technologische obstakels waaronder de displays van het apparaat zelf. \autocite{YunHan2018}
Hierdoor was het aan koplopers zoals Meta en Varjo om aanzienlijke budgetten te besteden om deze obstakels te kunnen overbruggen en de fundamenten van deze markt tot stand te krijgen.
Eerst kwamen zakelijke modellen zoals de Meta Quest Pro en Varjo XR-1, gevolgd door een influx aan consument-gerichte versies waaronder de Meta Quest 3 en de opkomende Apple VisionPro.

Tropos AR biedt een out of the box user interface (UI) aan voor mobiele AR-ontwikkelaars via haar Troposphere software-development kit.
Ontwikkelaars hoeven hierdoor niet langer te focussen op de UI, een groot onderdeel van wat mobiele AR maakt wat het is, waardoor ontwikkelingstijd aanzienlijk verkleind kan worden.
Echter, met de komst van steeds vaker gekochte consument-gerichte AR-apparaten ontstaat een nieuwe markt voor AR-ontwikkelaars.
Hoewel Tropos AR reeds knowhow opgebouwd heeft van bestaande UI-technieken binnen AR op smartphones, kunnen deze niet altijd even goed vertaald worden naar head-mounted AR.

Concreet is Tropos AR op zoek naar een nieuwe, meer intu\"{\i}tieve manier van interageren met de virtuele objecten die via AR in de echte wereld geprojecteerd worden.
Ook is er vraag naar de verschillende manieren om een menu af te beelden en hierbinnen te navigeren.
Uit dit onderzoek zal een rapport met aanbevelingen ontstaan voor Tropos AR.


%---------- Stand van zaken ---------------------------------------------------

\section{State-of-the-art}%
\label{sec:state-of-the-art}

Hier beschrijf je de \emph{state-of-the-art} rondom je gekozen onderzoeksdomein, d.w.z.\ een inleidende, doorlopende tekst over het onderzoeksdomein van je bachelorproef. Je steunt daarbij heel sterk op de professionele \emph{vakliteratuur}, en niet zozeer op populariserende teksten voor een breed publiek. Wat is de huidige stand van zaken in dit domein, en wat zijn nog eventuele open vragen (die misschien de aanleiding waren tot je onderzoeksvraag!)?

Je mag de titel van deze sectie ook aanpassen (literatuurstudie, stand van zaken, enz.). Zijn er al gelijkaardige onderzoeken gevoerd? Wat concluderen ze? Wat is het verschil met jouw onderzoek?

Verwijs bij elke introductie van een term of bewering over het domein naar de vakliteratuur, bijvoorbeeld~\autocite{Sutherland1968}! Denk zeker goed na welke werken je refereert en waarom.

Draag zorg voor correcte literatuurverwijzingen! Een bronvermelding hoort thuis \emph{binnen} de zin waar je je op die bron baseert, dus niet er buiten! Maak meteen een verwijzing als je gebruik maakt van een bron. Doe dit dus \emph{niet} aan het einde van een lange paragraaf. Baseer nooit teveel aansluitende tekst op eenzelfde bron.

Als je informatie over bronnen verzamelt in JabRef, zorg er dan voor dat alle nodige info aanwezig is om de bron terug te vinden (zoals uitvoerig besproken in de lessen Research Methods).

% Voor literatuurverwijzingen zijn er twee belangrijke commando's:
% \autocite{KEY} => (Auteur, jaartal) Gebruik dit als de naam van de auteur
%   geen onderdeel is van de zin.
% \textcite{KEY} => Auteur (jaartal)  Gebruik dit als de auteursnaam wel een
%   functie heeft in de zin (bv. ``Uit onderzoek door Doll & Hill (1954) bleek
%   ...'')

Je mag deze sectie nog verder onderverdelen in subsecties als dit de structuur van de tekst kan verduidelijken.

%---------- Methodologie ------------------------------------------------------
\section{Methodologie}%
\label{sec:methodologie}

Hier beschrijf je hoe je van plan bent het onderzoek te voeren. Welke onderzoekstechniek ga je toepassen om elk van je onderzoeksvragen te beantwoorden? Gebruik je hiervoor literatuurstudie, interviews met belanghebbenden (bv.~voor requirements-analyse), experimenten, simulaties, vergelijkende studie, risico-analyse, PoC, \ldots?

Valt je onderwerp onder één van de typische soorten bachelorproeven die besproken zijn in de lessen Research Methods (bv.\ vergelijkende studie of risico-analyse)? Zorg er dan ook voor dat we duidelijk de verschillende stappen terug vinden die we verwachten in dit soort onderzoek!

Vermijd onderzoekstechnieken die geen objectieve, meetbare resultaten kunnen opleveren. Enquêtes, bijvoorbeeld, zijn voor een bachelorproef informatica meestal \textbf{niet geschikt}. De antwoorden zijn eerder meningen dan feiten en in de praktijk blijkt het ook bijzonder moeilijk om voldoende respondenten te vinden. Studenten die een enquête willen voeren, hebben meestal ook geen goede definitie van de populatie, waardoor ook niet kan aangetoond worden dat eventuele resultaten representatief zijn.

Uit dit onderdeel moet duidelijk naar voor komen dat je bachelorproef ook technisch voldoen\-de diepgang zal bevatten. Het zou niet kloppen als een bachelorproef informatica ook door bv.\ een student marketing zou kunnen uitgevoerd worden.

Je beschrijft ook al welke tools (hardware, software, diensten, \ldots) je denkt hiervoor te gebruiken of te ontwikkelen.

Probeer ook een tijdschatting te maken. Hoe lang zal je met elke fase van je onderzoek bezig zijn en wat zijn de concrete \emph{deliverables} in elke fase?

%---------- Verwachte resultaten ----------------------------------------------
\section{Verwacht resultaat, conclusie}%
\label{sec:verwachte_resultaten}

Hier beschrijf je welke resultaten je verwacht. Als je metingen en simulaties uitvoert, kan je hier al mock-ups maken van de grafieken samen met de verwachte conclusies. Benoem zeker al je assen en de onderdelen van de grafiek die je gaat gebruiken. Dit zorgt ervoor dat je concreet weet welk soort data je moet verzamelen en hoe je die moet meten.

Wat heeft de doelgroep van je onderzoek aan het resultaat? Op welke manier zorgt jouw bachelorproef voor een meerwaarde?

Hier beschrijf je wat je verwacht uit je onderzoek, met de motivatie waarom. Het is \textbf{niet} erg indien uit je onderzoek andere resultaten en conclusies vloeien dan dat je hier beschrijft: het is dan juist interessant om te onderzoeken waarom jouw hypothesen niet overeenkomen met de resultaten.



    \printbibliography[heading=bibintoc]

\end{document}