%---------- Inleiding ---------------------------------------------------------

\section{Introductie}%
\label{sec:introductie}

Augmented reality, of kortweg AR, omvat een samensmelting van de virtuele wereld met de echte wereld door middel van een overlay.
Deze technologie bestaat voornamelijk in twee vormen: mobiele AR, die inmiddels al een wijde aanneming kent volgens een onderzoek van \textcite{Berggren2023}, en head-mounted AR.
Voor de voorgaande vorm biedt Tropos AR een out of the box user interface (UI) aan via haar Troposphere software-development kit.
Mobiele AR-ontwikkelaars hoeven hierdoor niet langer de focus te leggen op interacties van gebruikers met de AR-wereld, waardoor ontwikkelingstijd aanzienlijk verkleind kan worden.

Head-mounted Augmented Reality daarentegen kende meer technischere obstakels waardoor er tot recentelijk geen consument-gerichte modellen op de markt waren.
2023 kende echter een opkomst van dit soort modellen, onder de vorm van de Meta Quest 3 en de aankondiging van de Apple VisionPro.
Met de influx van deze head-mounted devices komt ook het ontstaan van een nieuwe markt waar Tropos AR op wilt inspelen.

Hoewel Tropos AR reeds knowhow opgebouwd heeft van bestaande UI-technieken binnen AR op smartphones, kunnen deze niet altijd even goed vertaald worden naar head-mounted AR.
Concreet is Tropos AR op zoek naar een manier om bestaande mobiele AR-ervaringen te vertalen naar deze nieuwe, opkomende vorm.
Ook is er vraag naar de verschillende manieren om een menu af te beelden en hierbinnen te navigeren.
Deze thesis probeert hier antwoorden voor te voorzien aan de hand van een rapport met aanbevelingen.

%---------- Stand van zaken ---------------------------------------------------

\section{Literatuurstudie}%
\label{sec:state-of-the-art}

\subsection{Het ontstaan van (mobiele) AR}
\label{subsec:wat-is-ar}
Met haar eerste iteratie in de vorm van een head-mounted display ontwikkeld door \textcite{Sutherland1968} , kende Augmented Reality een opmerkelijke voorsprong op de hedendaagse smartphone.
Toch kent deze vorm van AR niet dezelfde aanneming door het algemene publiek en is tot recentelijk enkel een vorm van mobiele AR (gebruikmakend van smartphones) aanwezig op de markt.

Head-mounted AR had vooral te lijden aan grotere technologische obstakels zoals de displays waarop een samensmelting van de virtuele met de echte wereld geprojecteerd wordt \autocite{YunHan2018} .
Hierdoor was het aan koplopers zoals Meta en Varjo om aanzienlijke budgetten te besteden om deze obstakels te kunnen overbruggen en de fundamenten van deze markt tot stand te krijgen. % TODO miss bron?
Intussen zagen smartphonemakers zoals Samsung en Apple een opportuniteit om deze technologie toch in de handen te zien krijgen bij consumenten, dit in de vorm van mobiele AR.
Zo bepalen hedendaagse smartphones met behulp van camera's en sensoren de plaats van digitale entiteiten en tonen ze deze vervolgens op hun scherm.

\subsection{Head-mounted AR en gebaren}
\label{subsec:ar-gestures}
Head-mounted AR is een radicaal andere ervaring van de samensmelting van virtueel met fysiek: in tegenstelling tot mobiele AR neemt deze vorm van AR heel het gezichtsveld in beslag.
Hierdoor verdwijnt echter ook de on-screen navigatie die men vandaag de dag gewend is aan smartphones en wordt deze vervangen met gebaren.

Een mogelijk alternatief hiervoor is om via gebaren te interageren met virtuele objecten, een alternatief waar Apple vooral op zal steunen met haar Vision Pro model.
De werking hiervan wordt bepaald door middel van camera's die de posities van de gebruikers' handen bepaalt en vertaalt naar interacties met menu's \textcite{Shrestha2018} .
Uit onderzoek van \textcite{Datcu2013} blijkt dat deze manier van virtuele objecten en menu's te besturen de voorkeur kent bij gebruikers.

\subsection{Het voordeel van head-mounted devices}
\label{subsec:benefits-hmd}
Head-mounted devices maken het mogelijk voor de gebruiker om direct aan de slag te gaan met de virtuele wereld zonder hierbij een smartphone vast te houden.
Het gevoel van kijken door een soort venster verdwijnt doordat de samensmeltende werelden direct geprojecteerd worden in de ogen van gebruikers.
Deze toepassing kent ook vandaag al een meerwaarde in onder andere de medische sector bij het opleiden van chirurgen \autocite{Waisberg2023} .

%---------- Methodologie ------------------------------------------------------
\section{Methodologie}%
\label{sec:methodologie}


Hier beschrijf je hoe je van plan bent het onderzoek te voeren. Welke onderzoekstechniek ga je toepassen om elk van je onderzoeksvragen te beantwoorden? Gebruik je hiervoor literatuurstudie, interviews met belanghebbenden (bv.~voor requirements-analyse), experimenten, simulaties, vergelijkende studie, risico-analyse, PoC, \ldots?

Valt je onderwerp onder één van de typische soorten bachelorproeven die besproken zijn in de lessen Research Methods (bv.\ vergelijkende studie of risico-analyse)? Zorg er dan ook voor dat we duidelijk de verschillende stappen terug vinden die we verwachten in dit soort onderzoek!

Vermijd onderzoekstechnieken die geen objectieve, meetbare resultaten kunnen opleveren. Enquêtes, bijvoorbeeld, zijn voor een bachelorproef informatica meestal \textbf{niet geschikt}. De antwoorden zijn eerder meningen dan feiten en in de praktijk blijkt het ook bijzonder moeilijk om voldoende respondenten te vinden. Studenten die een enquête willen voeren, hebben meestal ook geen goede definitie van de populatie, waardoor ook niet kan aangetoond worden dat eventuele resultaten representatief zijn.

Uit dit onderdeel moet duidelijk naar voor komen dat je bachelorproef ook technisch voldoen\-de diepgang zal bevatten. Het zou niet kloppen als een bachelorproef informatica ook door bv.\ een student marketing zou kunnen uitgevoerd worden.

Je beschrijft ook al welke tools (hardware, software, diensten, \ldots) je denkt hiervoor te gebruiken of te ontwikkelen.

Probeer ook een tijdschatting te maken. Hoe lang zal je met elke fase van je onderzoek bezig zijn en wat zijn de concrete \emph{deliverables} in elke fase?

%---------- Verwachte resultaten ----------------------------------------------
\section{Verwacht resultaat, conclusie}%
\label{sec:verwachte_resultaten}

Hier beschrijf je welke resultaten je verwacht. Als je metingen en simulaties uitvoert, kan je hier al mock-ups maken van de grafieken samen met de verwachte conclusies. Benoem zeker al je assen en de onderdelen van de grafiek die je gaat gebruiken. Dit zorgt ervoor dat je concreet weet welk soort data je moet verzamelen en hoe je die moet meten.

Wat heeft de doelgroep van je onderzoek aan het resultaat? Op welke manier zorgt jouw bachelorproef voor een meerwaarde?

Hier beschrijf je wat je verwacht uit je onderzoek, met de motivatie waarom. Het is \textbf{niet} erg indien uit je onderzoek andere resultaten en conclusies vloeien dan dat je hier beschrijft: het is dan juist interessant om te onderzoeken waarom jouw hypothesen niet overeenkomen met de resultaten.

